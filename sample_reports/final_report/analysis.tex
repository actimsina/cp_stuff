%!TEX root = final_report.tex
\chapter{Analysis specification} % (fold)
\label{cha:analysis}

Why we need to perform analysis? --Give adequate reasons for conducting analysis

Provide guide to the reader of your analysis document (Which section/chapter is where in your document, and what it consists of?) For example: The requirements engineering process is described in Section~\ref{sec:requirements}. The Use cases and architectures are provided in Section~\ref{sec:use_cases}, and~\ref{sec:architecture}.

This chapter should have following sections, along with its own introduction and conclusion:
\begin{itemize}
  \item Use cases
  \item Requirements
  \item Architecture
\end{itemize}

\section{Use cases} % (fold)
\label{sec:use_cases}
This section should provide:
\begin{itemize}
	\item The use cases that support the requirements of your project
	\item Use cases should have a use case diagram
	\item Each use case should have a scenario description
	\item For scenario description, you can use a suitable style
	\begin{itemize}
		\item Paragraph style
		\item Single column or sequence of interaction steps
		\item More formal styles with pre-conditions, post-conditions, entry criteria, exit criteria, etc.
	\end{itemize}
	\item You can break down your use case diagram based on major components or modules of your system. 
	\item You should use various relations between use cases whenever applicable. Most frequent relationships are: 
	\begin{itemize}
		\item extends
		\item includes
		\item uses
	\end{itemize}
\end{itemize}

One thing you need to have in your mind while developing use cases is ``Developing use case is fundamentally a scenario writing activity, not just drawing use case diagrams''(\cite{larman2012applying}). Hence a typical use case 
% section use_cases (end)

\section{Requirements} % (fold)
\label{sec:requirements}

How you have collected requirements for this project? Describe in detail your requirements capture, analysis, and specification method(s). 

\subsection{Functional requirements} % (fold)
\label{sub:functional_requirements}

You should at least have \textbf{10} functional requirements.

\hrule
\begin{description}
	\item[ID:] R1
	\item[Title:] User signup
	\item[Description:] A new user should be able to register through the web portal. The user must provide user-name, password and e-mail address.
	\item[Rational:] To acquire users credentials for login process
\end{description}
\hrule
\begin{description}
	\item[ID:] R2
	\item[Title:] User login
	\item[Description:] Existing user should be able to login to the system using existing user name and password. Appropriate message to the user should be provided whether one has entered correct credentials.
	\item[Rational:] To maintain user security and privacy
\end{description}
\hrule

% subsection functional_requirements (end)
\subsection{Non-functional requirements} % (fold)
\label{sub:non_functional_requirements}
This section should provide quality or non-functional requirements for your system. You should at least have 5 testable quality requirements. The non-functional requirements are:
\begin{description}
	\item[Performance] It should specify both the static and the dynamic performance requirements placed on the software or on human interaction with software as a whole. For example: the number of simultaneous users to be supported; amount and type of information to be handled, etc. All the requirements should be stated n measurable terms. For example: \emph{95\% of the transactions should be processed in less than 1s.}

	\item[Reliability] This should specify the factors required to establish the required reliability of the software system at time of delivery. For example: \emph{The system produces reliable results 99\% of the time}.

	\item[Availability] This should specify the factors required to guarantee a defined availability level for the entire system such as checkpoint, recovery, and restart. For example: \emph{The system should be 99\% available.}

	\item[Security] This should specify the factors that protect the software from accidental or malicious access, use, modification, destruction, or disclosure. Specific requirements in this area could include the need to:
	\begin{itemize}
		\item Utilize certain cryptographic techniques;
		\item Keep specific log or history data sets;
		% \item Assign certain functions to different modules;
		\item Restrict communications between some areas of the program;
		\item Check data integrity for critical variables;
	\end{itemize}

	Example non-functional requirement:
	\hrule
	\begin{description}
		\item[ID:] Q1
		\item[Title:] Communication security
		\item[Description:] The message should be encrypted for log-in communications, so others cannot get user-name and password from those messages.So all of the login messages should be encrypted.
		\item[Rational:] To maintain user secrecy and privacy
	\end{description}
	\hrule
	% \item[Maintainability]
	% \item[Portability]
\end{description}
% subsection non_functional_requirements (end)

\subsection{Prioritization} % (fold)
\label{sub:prioritization}

Use MoSCoW prioritization technique to prioritize your all requirements. Provide brief description of how you will perform MoSCow prioritization technique with relevant references.

Provide final result of prioritization in here.

For example (Just an example, but you are free to use your custom way to present the prioritization process):
\begin{table}[htb!]
  \centering
\begin{tabular}{l|c}
  \bf{Requirement} & \bf{MoSCoW}\\ \hline\hline
  R1. User Authentication & Should \\ \hline
  ... & ... \\ \hline
  ... & ... \\ \hline
\end{tabular}
  \caption{Requirements prioritization using MoSCoW}
  \label{tab:moscow}
\end{table}
% subsection prioritization (end)
% section requirements (end)

\section{Architecture} % (fold)
\label{sec:architecture}

\subsection{System architecture} % (fold)
\label{sub:system_architecture}
This section should have:

\begin{itemize}
	\item How you are going make your project easy to change and maintain?
	\item How you have separated different aspects (persistence, user interface, etc.) of your system?
	\item You should draw a system architecture diagram and explain it. 
	\item There are various ways you can build your system architecture, but two of the most widely used are Layered architecture and the MVC. Choose one and describe! I suggest you to use MVC.
\end{itemize}

% subsection system_architecture (end)

\subsection{Initial class diagram} % (fold)
\label{sub:initial_class_diagram}

This section should have high level class diagram of your application. You can come up with your initial class diagram using Natural Language Analysis method. The classes you identified in this section are at domain level. It should just have your classes identified from the requirements specification and use cases of your project. It should not consist of classes pertaining to databases, user interfaces, or any other implementation induced classes.

Your class diagram should depict:

\begin{itemize}
	\item Identified classes from Natural language analysis of your requirements specification
	\item Relevant attributes and methods of each class
	\item Relationships between those classes. You should at least identify \textbf{inheritance} and \textbf{composition} relationship between your classes.
	\item Finally, you should describe why and how you have come up with the class diagram.
\end{itemize}

% subsection initial_class_diagram (end)
% section architecture (end)

\section{Conclusion} % (fold)
Provide conclusion for your project analysis specification. It should summarize the activities you did in analyzing the system.
% section conclusion (end)
% chapter analysis (end)