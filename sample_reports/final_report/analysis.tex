%!TEX root = final_report.tex

\chapter{Analysis specification} % (fold)
\label{cha:analysis}

What activities are involved in analysis? --Describe in detail

Why we need to perform analysis? --Give adequate reasons for conducting analysis

Provide overview of your project, most probably with a overview diagram. Try to be creative here, and use your own suitable overview diagram such that even a non-technical person can understand it. Try to come up with a rich picture of your system.

Provide guide to the reader of your analysis document (Which section/chapter is where in your document, and what it consists of?) For example: The requirements engineering process is described in Section~\ref{sec:requirements}. The Use cases and architecture are provided in Section~\ref{sec:use_cases}, and~\ref{sec:architecture}. Finally, this report ends with a conclusion in a separate Section~\ref{sec:conclusion}.

This chapter should have following sections, along with its own introduction and conclusion:
\begin{itemize}
  \item Requirements
  \item Use cases
  \item Architecture
\end{itemize}

\section{Requirements} % (fold)
\label{sec:requirements}

How you have collected requirements for this project? Describe in detail your requirements capturing method. 

Put the same things from the last analysis specification document you have submitted along with the improvements.

\subsection{Functional requirements} % (fold)
\label{sub:functional_requirements}

You should at least have \textbf{10} functional requirements.

\hrule
\begin{description}
	\item[ID:] R1
	\item[Title:] User signup
	\item[Description:] A new user should be able to register through the web portal. The user must provide user-name, password and e-mail address.
	\item[Rational:] To acquire users credentials for login process
\end{description}
\hrule
\begin{description}
	\item[ID:] R2
	\item[Title:] User login
	\item[Description:] Existing user should be able to login to the system using existing user name and password. Appropriate message to the user should be provided whether one has entered correct credentials.
	\item[Rational:] To maintain user security and privacy
\end{description}
\hrule
\begin{description}
	\item[ID:] R3
	\item[Title:]
	\item[Description:] ...
	\item[Rational:] ...
\end{description}
% subsection functional_requirements (end)
\subsection{Non-functional requirements} % (fold)
\label{sub:non_functional_requirements}
This section should provide quality or non-functional requirements for your system. You should at least have 5 testable quality requirements. The non-functional requirements are:
\begin{description}
	\item[Performance] It should specify both the static and the dynamic performance requirements placed on the software or on human interaction with software as a whole. For example: the number of simultaneous users to be supported; amount and type of information to be handled, etc. All the requirements should be stated n measurable terms. For example: \emph{95\% of the transactions should be processed in less than 1s.}

	\item[Reliability] This should specify the factors required to establish the required reliability of the software system at time of delivery. For example: \emph{The system produces reliable results 99\% of the time}.

	\item[Availability] This should specify the factors required to guarantee a defined availability level for the entire system such as checkpoint, recovery, and restart. For example: \emph{The system should be 99\% available.}

	\item[Security] This should specify the factors that protect the software from accidental or malicious access, use, modification, destruction, or disclosure. Specific requirements in this area could include the need to:
	\begin{itemize}
		\item Utilize certain cryptographic techniques;
		\item Keep specific log or history data sets;
		% \item Assign certain functions to different modules;
		\item Restrict communications between some areas of the program;
		\item Check data integrity for critical variables;
	\end{itemize}

	Example non-functional requirement:
	\hrule
	\begin{description}
		\item[ID:] Q1
		\item[Title:] Communication security
		\item[Description:] The message should be encrypted for log-in communications, so others cannot get user-name and password from those messages.So all of the login messages should be encrypted.
		\item[Rational:] To maintain user secrecy and privacy
		\item[Dependencies:] R1, R2
	\end{description}
	\hrule
	% \item[Maintainability]
	% \item[Portability]
\end{description}
% subsection non_functional_requirements (end)

\subsection{Prioritization} % (fold)
\label{sub:prioritization}

Use MoSCoW prioritization technique to prioritize your all requirements. Provide brief description of how you will perform MoSCow prioritization technique with relevant references.

Provide final result of prioritization in here.

For example (Just an example, but you are free to use your custom way to present the prioritization process):
\begin{table}[htb!]
  \centering
\begin{tabular}{l|c}
  \bf{Requirement} & \bf{MoSCoW}\\ \hline\hline
  R1. User Authentication & Should \\ \hline
  ... & ... \\ \hline
  ... & ... \\ \hline
\end{tabular}
  \caption{Requirements prioritization using MoSCoW}
  \label{tab:moscow}
\end{table}
% subsection prioritization (end)
% section requirements (end)

\section{Use cases} % (fold)
\label{sec:use_cases}

One thing you need to have in your mind while developing use cases is ``Developing use case is fundamentally a scenario writing activity, not just drawing use case diagrams''(\cite{larman2012applying}).
% section use_cases (end)

\section{Architecture} % (fold)
\label{sec:architecture}

\subsection{System architecture} % (fold)
\label{sub:system_architecture}
Put the same things from the last analysis specification document you have submitted along with the improvements.
% subsection system_architecture (end)

\subsection{Initial class diagram} % (fold)
\label{sub:initial_class_diagram}
Put the same things from the last analysis specification document you have submitted along with the improvements.
% subsection initial_class_diagram (end)
% section architecture (end)

\section{Conclusion} % (fold)
Put the same things from the last analysis specification document you have submitted along with the improvements.
% section conclusion (end)
% chapter analysis (end)